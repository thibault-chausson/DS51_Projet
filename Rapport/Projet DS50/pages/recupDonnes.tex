%! Author = thibaultchausson
%! Date = 28/11/2022

%!TEX root = ../main.tex


\subsection{Importation des données}

Comme demandé dans le sujet, nous avons récupéré les données sur le site de la compétition \href{https://www.kaggle.com/competitions/imagenet-object-localization-challenge}{ImageNet Object Localization Challenge} de Kaggle. Nous avons ensuite extrait les données qui nous intéressent, à savoir les fichiers avec les identifiants suivant :
\begin{itemize}
    \item n02114367 (loup)
    \item n01484850 (requin)
    \item n01614925 (aigle)
    \item n02133161 (ours)
    \item n01537544 (passerin indigo)
    \item n01443537 (poisson rouge)
\end{itemize}


\subsection{Récupération des noms d'animaux}

\subsubsection{Récupération des identifiants des animaux sur Wikidata}

Pour ce faire, nous allons récupérer la liste des différents animaux existant sur Terre grace à WikiData avec une request SPARQL :

\begin{paddingTab} 
    \begin{customFrame}
    SELECT DISTINCT ?animal ?animalLabel WHERE {
      ?animal wdt:P31 wd:Q16521.
      ?animal rdfs:label ?animalLabel filter (lang(?animalLabel) = "en").
      #OPTIONAL {?animal wdt:P18 ?image.}
    } LIMIT 1000000
    \end{customFrame}
    \captionof{lstlisting}{Requête SPARQL pour récupérer les animaux \label{cs:animaux}}
\end{paddingTab}

Nous téléchargeons le fichier CSV pour les 1000000 premiers résultats et nous le sauvegardons dans le fichier \textit{animals.csv}, pour pouvoir l'exploiter plus tard.

\subsubsection{Récupération des identifiants des animaux sur Kaggle}

Sur Kaggle, nous récupérons le fichier \href{https://www.kaggle.com/competitions/imagenet-object-localization-challenge/data?select=LOC_synset_mapping.txt}{LOC\_synset\_mapping.txt}, qui contient la liste des identifiants des animaux avec leur nom.

La correspondance entre les 1000 synsets id et leurs descriptions. Par exemple, la ligne 1 indique n01440764 tanche, Tinca tinca signifie qu'il s'agit de la classe 1, que l'identifiant synset est n01440764 et qu'elle contient le poisson tanche.

\begin{customFrame}
n01440764 tench, Tinca tinca
n01443537 goldfish, Carassius auratus   
\end{customFrame}

\subsection{Récupération des données depuis une base de données : WikiData}

Pour ce faire, nous utiliserons la librairie \textit{wikibaseintegrator}\cite{wikibaseintegrator} qui permet d'interroger la base de données de WikiData.
Nous voulons récupérer l'ensemble des animaux de la base de données, pour ce faire, nous utiliserons : \textit{WikibaseIntegrator} wbi.item.get(entity\_id='Q729').


\subsection{Récupération des sous-classes des animaux}

Nous allons maintenant récupérer les sous-classes des animaux, nous avons choisi de les récupérer sur une profondeur de 2 niveaux. Pour cela, nous allons utiliser la requête SPARQL suivante :

\begin{paddingTab}
    \begin{customFrame}
        SELECT ?animal ?animalLabel ?subclass ?subclassLabel ?subclass2 ?subclass2Label
        WHERE
        {
          VALUES ?animal {wd:%s}
          ?animal wdt:P279 ?subclass .
          OPTIONAL { ?subclass wdt:P279 ?subclass2 . }
    
          SERVICE wikibase:label { bd:serviceParam wikibase:language "en". }
        }
    \end{customFrame}
    \captionof{lstlisting}{Requête SPARQL pour récupérer les sous-classes \label{cs:sClasse}}
\end{paddingTab}

\textit{{wd:\%s}} correspond à l'identifiant de l'animal dont nous voulons déterminer les sous-classes.

Nous avons récupéré la propriété \textit{subclass of}, au lieu de \textit{instant of}, car cette deuxième n'était pas toujours présente dans la liste des animaux fournis, contrairement à la première.

L'utilisation de la bibliothèque \textit{wikibaseintegrator} pour interroger la base de données de WikiData, nous permet de ne pas nous faire bloquer par l'API WikiData si nous faisons trop de requêtes.