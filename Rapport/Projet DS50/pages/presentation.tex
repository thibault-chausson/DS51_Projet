%! Author = thibaultchausson
%! Date = 28/11/2022

%!TEX root = ../main.tex

\subsection{Contexte du projet}

Nous avons réalisé ce projet dans le cadre de l'unité de valeur DS51 à l'UTBM, qui est une UV du bloc métier Data Science qui dote les étudiants de deux crédits en filière. 
Pour ce faire, nous sommes constitués en groupe de deux étudiants en informatique. 


Le projet se déroule durant la fin du semestre et est décomposé en un unique jalon. Le rendu final est constitué d'un rapport, d'une présentation et de l'ensemble des codes sources.




\subsection{Les objectifs du projet}


L'usage de l'apprentissage supervisé en Machine Learning offre aux ordinateurs, entre autres, la capacité d'apprendre à identifier automatiquement des objets grâce à des données étiquetées. 
Les résultats des algorithmes les plus avancés, comme les réseaux de neurones profonds, sont stupéfiants.

Cependant, ces algorithmes demandent un grand nombre d'exemples pour leur phase d'apprentissage. 
En outre, lorsqu'ils sont employés pour étiqueter de nouvelles données, ils ne peuvent pas expliquer le raisonnement derrière leur décision. 
Ils peuvent cependant fournir un score qui représente la probabilité que leur décision soit correcte, ou indiquer la portion de la donnée qui a influencé leur décision.

Dans ce projet, nous supposons que les ontologies peuvent en partie surmonter ces deux obstacles scientifiques.